
\chapter{Application Code Security}

\section{System Definition}

This is a link \cite{devsecops}.

Code Scanning can be divided into the following use cases

\refstepcounter{usecaseid}
\usecase{%
  id             = {\theusecaseid},
  title          = {Code Scanning Alert Triage},
  description    = {%
      As a commiter with write access to \textit{reponame}, I want to be able to remediate alerts on my pull request
    },
  actors         = {Developer},
  preconditions  = {%
      Pull request must be raised and code scanning complete.
    },
  postconditions = {%
      All alerts on PR are resolved.
    },
  basicFlow      = {
      \item The user is notified of alerts
      \item The user remediates alert by pushing in a fix to the Pull Request
      \item This triggers the code scanning action
      \item The alert is resolved
    },
  alternateFlow  = {
      The user is notified of alerts
      The user dismisses alert as false positive
      The alert is resolved
    }
}

\section{Vulnerability Discovery}
\label{sec:vuln-disco}

\lipsum[1-2]

% @formatter:off
\begin{minted}{c}
#include <stdio.h>

int main(void) {
    printf("Hello World!");
    return 0;
  }
\end{minted}
% @formatter:on


\lipsum[1-2]

\section{Exploitation}
\section{References}
\printbibliography[heading=none]



